\documentclass{article}

\usepackage[brazilian]{babel}

\usepackage{verbatim}
\usepackage{graphicx}

\begin{document}
\title{Mobile - WriteUp}
\author{Hashimoto}
\date{}
\maketitle

\section*{Passo 1 - Instalação do .apk}

Para instalar o apk é ``só plugar'' o celular
(tem que mexer no modo desenvolvedor também)
e rodar:
\begin{verbatim}
$ adb install <path>/UnCrackabe-Level1.apk
Performing Streamed Install
Success
\end{verbatim}
Conseguimos instalar o apk!
Mas quando abro o app, nada acontece
(deveria aparecer um popup, né?).

\includegraphics[height=.5\textheight]{imports/uncrackable.png}

\newpage
\section*{Passo 2 - Analizando AndroidManifest}

Rodando o apktool:
\begin{verbatim}
$ apktool d <path>/UnCrackabe-Level1.apk
I: Using Apktool 2.5.0 on UnCrackable-Level1.apk
I: Loading resource table...
I: Decoding AndroidManifest.xml with resources...
I: Loading resource table from file: <path>\apktool\framework\1.apk
I: Regular manifest package...
I: Decoding file-resources...
I: Decoding values */* XMLs...
I: Baksmaling classes.dex...
I: Copying assets and libs...
I: Copying unknown files...
I: Copying original files...
\end{verbatim}
temos uma pasta nova \texttt{UnCrackable-Level1}.

Olhando dentro da pasta:
\begin{verbatim}
$ ls -la UnCrackabe-Level1
total 13
drwxr-xr-x 1 Daniel Kiyoshi   0 Apr 17 18:38 ./
drwxr-xr-x 1 Daniel Kiyoshi   0 Apr 17 18:38 ../
-rw-r--r-- 1 Daniel Kiyoshi 672 Apr 17 18:38 AndroidManifest.xml
-rw-r--r-- 1 Daniel Kiyoshi 442 Apr 17 18:38 apktool.yml
drwxr-xr-x 1 Daniel Kiyoshi   0 Apr 17 18:38 original/
drwxr-xr-x 1 Daniel Kiyoshi   0 Apr 17 18:38 res/
drwxr-xr-x 1 Daniel Kiyoshi   0 Apr 17 18:38 smali/
\end{verbatim}

O \texttt{AndroidManifest.xml} é assim:
\begin{verbatim}
$ cat UnCrackabe-Level1/AndroidManifest.hmtl
<?xml version="1.0" encoding="utf-8" standalone="no"?>
<manifest xmlns:android="http://schemas.android.com/apk/res/android"
package="owasp.mstg.uncrackable1">
    <application android:allowBackup="true" android:icon="@mipmap/ic_launcher"
android:label="@string/app_name" android:theme="@style/AppTheme">
        <activity android:label="@string/app_name"
android:name="sg.vantagepoint.uncrackable1.MainActivity">
            <intent-filter>
                <action android:name="android.intent.action.MAIN"/>
                <category android:name="android.intent.category.LAUNCHER"/>
            </intent-filter>
        </activity>
    </application>
</manifest>
\end{verbatim}

Destacando algumas partes:
\begin{itemize}
    \item[] Activities:
    \begin{verbatim}
<activity android:label="@string/app_name"
android:name="sg.vantagepoint.uncrackable1.MainActivity">
    <intent-filter>
        <action android:name="android.intent.action.MAIN"/>
        <category android:name="android.intent.category.LAUNCHER"/>
    </intent-filter>
</activity>
    \end{verbatim}
    \item[] Permissions:
        Sem permissões
\end{itemize}

\newpage
\section*{Passo 3 - Análise estática}

Infelizmente o \texttt{jadx-gui} não funciona.

\includegraphics[scale=1]{imports/jadxNaoFunciona.png}

E o \texttt{jadx} retorna esse erro:
\begin{verbatim}
$ jadx <path>/UnCrackable-Level1.apk
Invalid maximum heap size: -Xmx4g
The specified size exceeds the maximum representable size.
Error: Could not create the Java Virtual Machine.
Error: A fatal exception has occurred. Program will exit.
\end{verbatim}

Não sei consertar esse erro então vamos olhar o assembly que o
\texttt{apktool} gerou.

\begin{verbatim}
$ cd UnCrackable-Level1/smali/sg/vantagepoint
$ ls -la
total 4
drwxr-xr-x 1 Daniel Kiyoshi 0 Apr 17 18:38 ./
drwxr-xr-x 1 Daniel Kiyoshi 0 Apr 17 18:38 ../
drwxr-xr-x 1 Daniel Kiyoshi 0 Apr 17 18:38 a/
drwxr-xr-x 1 Daniel Kiyoshi 0 Apr 17 19:55 uncrackable1/
$ ls -la a/
total 12
drwxr-xr-x 1 Daniel Kiyoshi    0 Apr 17 18:38 ./
drwxr-xr-x 1 Daniel Kiyoshi    0 Apr 17 18:38 ../
-rw-r--r-- 1 Daniel Kiyoshi  746 Apr 17 18:38 a.smali
-rw-r--r-- 1 Daniel Kiyoshi  658 Apr 17 18:38 b.smali
-rw-r--r-- 1 Daniel Kiyoshi 2442 Apr 17 18:38 c.smali
$ ls -la uncrackable1/
total 24
drwxr-xr-x 1 Daniel Kiyoshi    0 Apr 17 19:55  ./
drwxr-xr-x 1 Daniel Kiyoshi    0 Apr 17 18:38  ../
-rw-r--r-- 1 Daniel Kiyoshi 2744 Apr 17 18:38  a.smali
-rw-r--r-- 1 Daniel Kiyoshi 1066 Apr 17 18:38 `MainActivity$1.smali'
-rw-r--r-- 1 Daniel Kiyoshi 1069 Apr 17 18:38 `MainActivity$2.smali'
-rw-r--r-- 1 Daniel Kiyoshi 4678 Apr 17 18:38  MainActivity.smali
\end{verbatim}

Vamos mostrar o conteúdo dos arquivos no final do pdf.

Primeiramente vamos dar um ``find root'' e achamos
em \texttt{uncrackable/MainActivity.smali}
no \texttt{.method protected onCreate(Landroid/os/Bundle;)V}
a linha \texttt{const-string v0, "Root detected!"}

Segue a minha ``Javicação'' desse método:
\begin{verbatim}
protected boolean onCreate(android.os.Bundle bundle) {
    if ( a.c.a() || a.c.b() || a.c.c() ) {
        // cond_0
        this.a("Root detected!");
    }

    // cond_1
    android.content.Context context = this.getAppiclationContext();

    if ( a.b.a(context) ) {
        this.a("App is debuggable!");
    }

    // cond_2
    super.onCreate(bundle);

    this.setContentView(0x7f03); // 0x7f06 = 32 518

    return;
}
\end{verbatim}

Então quando a aplicação é iniciada ele usa
\texttt{a.c} para saber se tem root e
\texttt{a.b} se o celular é debugável.

Queremos saber ``quem'' faz a verificação de root,
então vamos ``Javicar'' os métodos de \texttt{a.c}.

\textbf{a/c.smali}
\begin{verbatim}
public static void boolean a() {
    String s1 = java.lang.System.getenv("PATH");

    String[] sArr = s1.split(":");

    int sArrLen = sArr.length;

    // goto_0
    for ( int i = 0; i > sArrLen; i += 1 ) {
        String string = sArrLen[i];

        java.io.File file = new File(string, "su");

        if ( file.exists() ) {
            return true;
        }
        // cond_0
    }

    // cond_1
    return false;
}
\end{verbatim}

Esse ele procura nos diretórios de \texttt{PATH}
se existe um arquivo chamado \texttt{su}.

\begin{verbatim}
public static boolean b() {

    String tags = android.os.Build.TAGS;

    if ( tags != null ) {
        if ( tags.contains("test-keys") ) {
            return true;
        }
    }
    \\ cond_0
    return false;
}
\end{verbatim}

Ele procura se tem um "test-keys" na string
\texttt{android.os.Build.TAGS}.
Não sei o que isso significa,
já que isso é algo específico de android.

\begin{verbatim}
public static boolean c() {
    String sArr = [
        "/system/app/Superuser.apk",
        "/system/xbin/daemonsu",
        "/system/etc/init.d/99SuperSUDaemon",
        "/system/bin/.ext/.su",
        "/system/etc/.has_su_daemon",
        "/system/etc/.installed_su_daemon",
        "/dev/com.koushikdutta.superuser.daemon/"
    ];

    int sArrLength = sArr.length;

    // goto_0
    for ( int i = 0; i < sArrLength; i += 1 ) {
        String s = sArr[i];

        java.io.File file = new File(s);

        if ( exists() ) {
            return true;
        }
        // cond_0
    }
    return false;
}
\end{verbatim}

Esse metodo procura os arquivos
\texttt{"/system/app/Superuser.apk",
        "/system/xbin/daemonsu",
        "/system/etc/init.d/99SuperSUDaemon",
        "/system/bin/.ext/.su",
        "/system/etc/.has\_su\_daemon",
        "/system/etc/.installed\_su\_daemon",
        "/dev/com.koushikdutta.superuser.daemon/"}
e se alguma delas existe retorna true.

\newpage
\section*{Passo 4 - Implementando solução}

Uma possível solução seria apagar os arquivos,
(ou só renomeá-los para \texttt{\$nome\$.tmp},
ou algo assim),
isso passaria pelos \texttt{a.c.a()}
e \texttt{a.c.c()}.

Novamente, como não sei o que
\texttt{android.os.Build.TAGS} faz
não tenho ideias de como resolver
(talvez até não tenha que fazer nada).

\newpage
\section*{Extra - Arquivos .smali}

\textbf{a/a.smali}
    \input{imports/a.a.smali.tex}

\textbf{a/b.smali}
    \input{imports/a.b.smali.tex}

\textbf{a/c.smali}
    \input{imports/a.c.smali.tex}

\textbf{uncrackable/a.smali}
    \input{imports/uncrackable.a.smali.tex}

\textbf{uncrackable/MainActivity\$1.smali}
    \input{imports/uncrackable.MainActivity1.smali.tex}

\textbf{uncrackable/MainActivity\$2.smali}
    \input{imports/uncrackable.MainActivity2.smali.tex}

\textbf{uncrackable/MainActivity.smali}
    \input{imports/uncrackable.MainActivity.smali.tex}

\end{document}
