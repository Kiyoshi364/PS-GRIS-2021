\documentclass{article}

\usepackage[brazilian]{babel}

\begin{document}
\title{Pequeno Texto sobre Limites (ou a Inexistência deles) da Ética Utilitarismo para o Ethical Hacking}
\author{Hashimoto}
\date{}
\maketitle

Dado que a definição de \emph{Ethical Hacking} usada para esse texto é:
``é o ato de tentar hackear sistemas para fins de descobrir
vulnerabilidades para que estas sejam corrigidas.''
Em outras palavras,
seria algo parecido a \emph{cutucar} uma \emph{fechadura}
para saber se ela vai abrir com \emph{outra coisa}
que não a \emph{chave correta};
se abrir com essa \emph{outra coisa},
ajudar a quem faz a fechadura
a fazer uma nova que não abra com a \emph{outra coisa}.
Venho perguntar:
\emph{em que momentos é bom/certo cutucar a fechadura e
em que momentos é ruim/errado?}

O bom e ruim, o certo e errado
não são coisas muito bem definidas e
muitas vezes acabam dependendo de várias coisas.
Então parece uma boa ideia pedir uma ajuda para filosofia,
nesse caso, utilitarista.
Tirando da Wikipedia:
``O utilitarismo é uma família de teorias consequencialistas [...]
que afirma que as ações são boas quando
tendem a promover a felicidade
e más quando tendem a promover o oposto da felicidade.''

``Olha que legal! Isso quer dizer que,
segundo o utilitarismo,
podemos sair poraí cutucando fechaduras
e avisar o dono delas o que aconteceu;
porque, no final,
fechaduras melhores deixam pessoas mais felizes?''
Mais ou menos... no geral, sim?
Para garantir que é verdade,
deve ser mostrado que, de alguma forma
``cutucar fechaduras'' e ``avisar o dono que a fechadura é ruim''
causam menos \emph{infelicidade} (o ``oposto de felicidade'')
que ``ter fechaduras melhores'' causam \emph{felicidade}.
Como se mostra isso?
Não sei, até o momento em que me mostrarem uma prova geral,
cada caso é um caso.

O que poderia dar errado para que isso não soasse tão óbvio,
pelo menos para o autor desse texto?
Por exemplo, em uma situação que
cutucar fechaduras poderia causar algum dano ou quebrar a fechadura.

``Eu não tenho recursos para comprar/consertar uma fechadura.
Se alguém cutucar e essa fechadura quebrar,
eu não vou conseguir comprar outra e vou ficar infeliz até esse momento.
Entretanto muita gente usa a mesma fechadura e
se sentiria um pouco mais segura
se soubesse que a sua fechadura tem um defeito,
mas ela pode ser trocada por uma fechadura melhor.''

A possível ``quantidade de infelicidade'' minha
é pequena o suficiente para justificar a possível
``quantidade de felicidade'' somada dessa gente toda?
Vamos supor que sim.
Se supomos que sim,
agora todo mundo pode ter a sua fechadura cutucada
com uma chance de quebrar,
isso causaria um mínimo de preocupação nessas pessoas.
Vamos botar essa quantidade de infelicidade na conta.
E agora? Existe outra consequência disso?
A quantidade de felicidade continua maior?
Se foram dois ``sim''s e eu não me esqueci de nada,
\emph{parabéns} podemos cutucar as fechaduras dos outros, \emph{uhuuul}!
Mas se teve algum não, não temos essa certeza.
\end{document}
