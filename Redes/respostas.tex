\documentclass{article}

\usepackage[brazilian]{babel}

\usepackage{amsmath}

\begin{document}
\title{Redes}
\author{Hashimoto}
\date{}
\maketitle

\section*{Questão 1}

\begin{enumerate}
    \item Física:
        Transferência de dados brutos, converte sinais em bits e
        vise-versa e os envia, não faz correção de erros.
    \item Enlace:
        Conecta 1 e 3, usa MAC, estabelece protocolo utilizado,
        pode corrigir erros.
    \item Rede:
        Endereça pacotes, converte IP em MAC, decide se o pacote
        deve ser roteado ou não.
    \item Transporte:
        Organiza pacotes (junta na ordem certa), pode se orientar
        de acordo com o protocolo.
    \item Sessão:
        Estabelece uma forma de transmitir dados entre duas
        aplicações, corrigue erros.
    \item Apresentação:
        Comprime e/ou Criptografa os dados.
    \item Aplicação:
        Escolhe aplicações e protocolos para serem usados.
\end{enumerate}

\section*{Questão 2}\

Um domínio de broadcast é um segmento lógica de uma rede de computadores em
que um aparelho conectado à rede é capaz de se comunicar com outro aparelho
conectado à rede sem a necessidade de utilizar um dispositivo de roteamento.

Um domínio de colisão é uma área lógica onde os pacotes podem colidir uns
contra os outros.

\newpage
\section*{Questão 3}

\subsection*{Domínios de Rede/Broadcast:}
\begin{itemize}
    \item PC1, Switch S1, PC2, Roteador R1
    \item PC3, Roteador R1
    \item Roteador R1, Roteador R2
    \item PC4, HUB H1, PC5, Roteador R2
\end{itemize}

\subsection*{Domínios de Colisão:}
\begin{itemize}
    \item PC1, Switch S1
    \item PC2, Switch S1
    \item Switch S1, Roteador R1
    \item Roteador R1, PC3
    \item Roteador R1, Roteador R2
    \item PC4, HUB H1, PC5, Roteador R2
\end{itemize}

\section*{Questão 4}

\begin{tabular}{|c|c|c|c|c|}
    \hline
    N & MAC ORIG & IP ORIG & MAC DEST & IP DEST \\
    \hline
    1 & A         & A        & R1        & B        \\
    2 & R1        & A        & R2        & B        \\
    3 & R2        & A        & B         & B        \\
    4 & B         & B        & R2        & A        \\
    5 & A         & B        & R1        & A        \\
    6 & R1        & B        & A         & A        \\
    \hline
\end{tabular}

\begin{enumerate}
    \item A manda mensagem para o R1.
    \item R1 manda mensagem para R2.
    \item R2 manda mensagem para B.
    \item B manda ACK para R2.
    \item R2 manda ACK para R1.
    \item R1 manda ACK para A.
\end{enumerate}

\newpage
\section*{Questão 5}

\begin{tabular}{|c|c|c|c|c|}
    \hline
    N & MAC ORIG & IP ORIG & MAC DEST & IP DEST \\
    \hline
    1 & A         & A        & R1        & B        \\
    2 & R1        & R1       & R2        & B        \\
    3 & R2        & R1       & B         & B        \\
    4 & B         & B        & R2        & R1       \\
    5 & A         & B        & R1        & R1       \\
    6 & R1        & B        & A         & A        \\
    \hline
\end{tabular}

\begin{enumerate}
    \item A manda mensagem para o R1.
    \item R1 manda mensagem para R2.
    \item R2 manda mensagem para B.
    \item B manda ACK para R2.
    \item R2 manda ACK para R1.
    \item R1 manda ACK para A.
\end{enumerate}

\section*{Questão 6}\

Segundo RFC793:
\texttt{A} manda um \texttt{Syn} para \texttt{B}
a fim de começar uma conexão, então \texttt{B}
manda um \texttt{ACK} (confirmando o \texttt{Syn}) e um \texttt{Syn},
e assim \texttt{A} responde com um \texttt{ACK} para
o \texttt{Syn} e a conexão é estabelecida.

\section*{Questão 7}\

\texttt{MDI} e \texttt{MDI-X} são formas
de ordenar as entradas dos fios
(que estão dentro dos cabos),
a diferença entre elas é que
\texttt{TX +} e \texttt{-} trocam de lugar com
\texttt{RX +} e \texttt{-}, e vice-versa.

\section*{Questão 8}

\begin{itemize}
    \item \texttt{A  <-> S1}: Straight
    \item \texttt{S1 <-> S2}: Cross
    \item \texttt{S2 <-> R1}: Straight
    \item \texttt{R1 <-> R2}: Cross
    \item \texttt{R2 <->  B}: Cross
\end{itemize}

\newpage
\section*{Questão 9}

\begin{enumerate}
    \item Broadcast \\
        Rede: 177.32.168.216 \\
        Host: 177.32.168.217 - 177.32.168.212 \\
        Broadcast: 177.32.168.223

    Cálculos: \\
        223 = 0b 1101 1111 \\
        248 = 0b 1111 1000 \\
        1101 1000 = 216

    \item Host \\
        Rede: 204.20.128.0 \\
        Host: 204.20.128.1 - 204.10.191.254  \\
        Broadcast: 204.10.191.255

    Cálculos: \\
        18 = 8+8+2 \\
        143 = 0b 1000 1111 \\
        1000 0000 = 128 \\
        1011 1111 = 191

    \item Host \\
        Rede:  36.72.0.0 \\
        Host:  36.72.0.1 - 36.73.255.254 \\
        Broadcast: 36.73.255.255

    \item Rede \\
        Rede:  7.26.0.64 \\
        Host:  7.26.0.65 - 7.26.0.126 \\
        Broadcast: 7.26.0.127

    Cálculos: \\
        26 = 8+8+8+2 \\
        64 = 0b 0100 0000

    \item Broadcast \\
        Rede:  200.201.173.184 \\
        Host:  200.201.173.183 - 200.201.173.186 \\
        Broadcast: 200.201.173.187

    Cálculos: \\
        252 = 0b 1111 1100 \\
        187 = 0b 1011 1011 \\
        1011 1000 = 184
\end{enumerate}

\newpage
\section*{Questão 10}

\begin{enumerate}
    \item Mesma rede \\
        224 = 0b 111 0 0000 \\
        154 = 0b 100 1 1010 \\
        158 = 0b 100 1 1110 \\

    \item Mesma rede \\
        248 = 0b 1111 1 000 \\
        142 = 0b 1000 1 110 \\
        137 = 0b 1000 1 001 \\

    \item Mesma Rede \\
        10 = 8+2 \\ .\hspace{1.75em}
             0b 11 00 0000 \\
        45 = 0b 00 10 1010 \\
        12 = 0b 00 00 1100 \\
\end{enumerate}

 \textit{
Obs: Os números antes o byte onde a máscara corta são iguais, por isso não é
necessário verificar.
}

\newpage
\section*{Questão 11}\

\begin{table}[h]
    \centering
    \label{Cheat Table 11}
    {Cheat Table \\ \quad \\}
    \begin{tabular}{|ccccc|}
        \hline
        $ 2^{ 2} -2 $&$ = $&$    2 $&$ \rightarrow $&$ /30 $\\
        $ 2^{ 3} -2 $&$ = $&$    6 $&$ \rightarrow $&$ /29 $\\
        $ 2^{ 4} -2 $&$ = $&$   14 $&$ \rightarrow $&$ /28 $\\
        $ 2^{ 5} -2 $&$ = $&$   30 $&$ \rightarrow $&$ /27 $\\
        $ 2^{ 6} -2 $&$ = $&$   62 $&$ \rightarrow $&$ /26 $\\
        $ 2^{ 7} -2 $&$ = $&$  126 $&$ \rightarrow $&$ /25 $\\
        $ 2^{ 8} -2 $&$ = $&$  254 $&$ \rightarrow $&$ /24 $\\
        $ 2^{ 9} -2 $&$ = $&$  510 $&$ \rightarrow $&$ /23 $\\
        $ 2^{10} -2 $&$ = $&$ 1022 $&$ \rightarrow $&$ /22 $\\
        $ 2^{11} -2 $&$ = $&$ 2046 $&$ \rightarrow $&$ /21 $\\
        \hline
    \end{tabular}
\end{table}

Olhando nossa \textit{Cheat Table}, precisamos de:
\begin{itemize}
    \item \textbf{25 hosts:} /27 (sobram 5)
    \item \textbf{60 hosts:} /26 (sobram 2)
    \item \textbf{120 hosts:} /25 (sobram 6) [x2]
    \item \textbf{500 hosts:} /23 (sobram 10) [x2]
\end{itemize}

Podemos dividir nosso /8 em /9, /10, /11, \(\dots\) /20, /21, /22 e /22.
Vamos usar o primeiro /22 para gerar os dois /23 que precisamos.
Vamos usar o segundo /22 para gerar dois /25, um /26 e um /27
(e sobram /27, /25, /23).

Segue um pequeno mapa: \\
\begin{verbatim}
/8->...->/21 -> /22 -> /23                      => 187.0.0.0   - 187.0.1.255
           |      L--> /23                      => 187.0.2.0   - 187.0.3.255
           L--> /22 -> /23 -> /24 -> /25        => 187.0.4.0   - 187.0.4.127
                         |      L--> /25        => 187.0.4.128 - 187.0.4.255
                         L--> /25 -> /26        => 187.0.5.0   - 187.0.5.63
                                L--> /26 -> /27 => 187.0.5.64  - 187.0.5.95
\end{verbatim}

Considerando apenas os endereços que precisamos:
\begin{itemize}
    \item \textbf{/23}: 187.0.0.0 - 187.0.1.225
    \item \textbf{/23}: 187.0.2.0 - 187.0.3.225
    \item \textbf{/25}: 187.0.4.0 - 187.0.4.127
    \item \textbf{/26}: 187.0.5.0 - 187.0.5.63
    \item \textbf{/27}: 187.0.5.64 - 187.0.5.95
\end{itemize}

\newpage
\section*{Questão 12}\

RIP e OSPF são Interior Gateway Protol (IGP) e BGP é de Exterior Gateway
Protocol (EGP).

\section*{Questão 13}\

Convertendo para unidade padrão:
\begin{gather*}
    64KB = 64 * 1024 * 8 = 524 \; 288 \; bits \\
    32KB = 32 * 1024 * 8 = 262 \; 144 \; bits \\
    15ms = 0.015s
\end{gather*}

Se todos mandam pacotes ao mesmo tempo:
\begin{gather*}
    524 \; 288 \; bits * 3 + 262 \; 144 \; bits * 2 \\
    1 \; 626 \; 864 \; bits + 524 \; 288 \; bits \\
    2 \; 151 \; 152 \; bits
\end{gather*}

Dividindo pela latência:
\begin{gather*}
    2 \; 151 \; 152 \; bits / 0.015 \; s \\
    143 \; 410 \; 133, 333 \dots \; bits/s
\end{gather*}

Dividindo por \(8\) ou \(1024\) para mudar a unidade:
\begin{gather*}
    143 \; 410 \; 133, 333 \dots \; bits/s \\
     17 \; 926 \; 266, 666 \dots \; B/s \\
     17 \; 506, 119 \; 791 \; 666 \dots \; KB/s \\
     17, 095 \; 820 \; 109 \; 049 \dots \; MB/s
\end{gather*}

\newpage
\section*{Questão 14}

\begin{itemize}
    \item \textbf{Sequence Number}:
        É o número do 1o byte da mensagem
        que está sendo enviado nesse pacote.
    \item \textbf{Acknowledment}:
        É o número do próximo byte que espera receber,
        confirmando o recebimento de todos os anteriores.
    \item \textbf{Window Size}:
        O tamanho da janela,
        quantidade de pacotes que são enviados por vez.
    \item \textbf{Flags}:
        \begin{itemize}
            \item URG:
                A mensagem é urgente,
                então deve ser enviada para a aplicação imediatamente.
            \item ACK:
                Marca que o campo ACK contém informação válida.
            \item PSH:
                Recebe prioridade (menor que urgente)
                para o tratamento da mensagem.
            \item RST:
                Encerra abruptamente a conexão nos dois lados.
            \item SYN:
                Tenta iniciar uma conexão.
            \item FIN:
                Encerra a conexão normalmente por um lado.
    \end{itemize}
\end{itemize}

\section*{Questão 15}\

Uma mensagem TCP é dividida em vários pacotes e
o número de sequenciamento de cada pacote é
o mesmo do 1o byte enviado nesse pacote.
Considerando Window Size como 1:

Alice manda um pacote de sequencia $X$ e tamanho $N$ para Bob.
Bob recebe e manda um pacote com ACK $X+N$
Alice manda o próximo pacote de sequencia $X+N$ e tamanho $M$
Bob recebe e manda um pacote com ACK $X+N+M$

\section*{Questão 16}\

Quando o timeout expira sem receber um ACK correspondente,
retransmite o pacote e redefine o tamanho da janela para 1,
voltando para \textit{slow start}.

\section*{Questão 17}\

O fast retransmit do TCP padrão funciona
retransmitindo o pacote quando recebe um ACK duplicado 3 vezes
(total de 4 ACKs iguais).

\newpage
\section*{Questão 18}

\begin{itemize}
    \item \textbf{Slow Start}:
        Funciona enquando \(cwnd < threshold\),
        \(cwnd\) duplica a cada ACK recebido.
    \item \textbf{Congestion Avoidance}:
        Funciona quando \(cwnd >= threshold\),
        \(cwnd\) aumenta em 1 a cada ACK recebido.
    \item Em caso de 3 ACKs duplicados é feito
        \(threshold := cwnd / 2\) e \(cwnd := threshold\).
    \item Em caso de \textit{timeout} é feito
        \(threshold := cwnd / 2\) e \(cwnd := 1\).
        \textit{(Obs: estamos em slow start.)}
\end{itemize}

\section*{Questão 19}\

O comportamento serrilhado ocorre porque
quando está em modo \textit{Congestion Avoidance},
o tamanho da janela vai crescendo linearmente e
quando ocorre um erro cai para metade.

Isso é importante para TCP, pois esse comportamento é
simples de ser implementado e
busca maximizar a quantidade de dados transmitidos,
evitando os congestionamentos.

\section*{Questão 20}\

\begin{itemize}
    \item \textbf{A} envia os pacotes 1 até 8 para \textbf{B}.
    \item \textbf{B} recebe os pacotes 2 até 9 em qualquer ordem,
        e manda seus respectivos ACKs para \textbf{A}.
    \item ACKs 2 e 3 se perdem.
    \item \textbf{A} recebe ACKs 4 até 9 em qualquer ordem.
    \item \textbf{A} envia os próximos pacotes 9 até 17 para \textbf{B}
        \textit{(Obs: tamanho da janela aumentou em 1)}.
\end{itemize}

\section*{Questão 21}\

Um Sistema Autônomo é um conjunto de redes, ou uma única rede, que além de
estar sob uma gestão comum tem características e políticas de roteamento
comuns.

\newpage
\section*{Questão 22}

\begin{table}[h]
    \centering
\begin{tabular}{|c|c|c|}
    \hline
    & Requisição & Resposta \\ \hline
    MAC DEST & 11111111 & MAC A \\ \hline
    MAC ORIG & MAC A & MAC B \\ \hline
    Controle & controle & controle \\ \hline
    Cabeçalho & & \\ \hline
    IP DEST & IP B & IP B \\ \hline
    MAC DEST & ? & ? \\ \hline
    ORIG & IP A & IP A \\ \hline
    MAC ORIG & MAC A & MAC A \\ \hline
    CRC & CRC & CRC \\ \hline
\end{tabular}
\end{table}

\section*{Questão 23}\

\textit{Carrier Sense Multiple Access with Collision Detection}
é um algoritmo para prevenir,
detectar e tratar colisões em redes Ethernet.

Algoritmo:
\begin{verbatim}
    Mandar Frame:
    . Se não tem frame pronto, espera ter um frame pronto.
    . Se o meio está ocupado, espera ficar desocupado.
    . Começa a transmitir o frame e monitore para detectar colisão.
    . Se ocoreu colisão, pule para :Colisão Detectada.
    . Resete `contador de retransmissão' e complete a transmissão do frame.
\end{verbatim}
\begin{verbatim}
    Colisão Detectada:
    . Continue a transmissão, mas com o `Jam Signal' (*) até um tempo mínimo.
    . Incremete `contador de retransmissão'.
    . Se `contador de retransmissão' >= `Máximo de Tentativas', aborte.
    . Calcule um número aleatório e espere esse tempo.
    . Tente transmitir o frame de novo.
\end{verbatim}

\texttt{(*)}\textit{:
    isso é para avisar aos outros da rede que ocorreu colisão.
}

\newpage
\section*{Questão 24}\

Encapsulamento é quando uma coisa A, engloba uma coisa B e pode ou não
adicionar outras coisas (geralmente adiciona). Em redes, acontece quando uma
informação passa de uma camada para outra: a informação da camada mais alta
passa a ser um campo para a camada mais baixa.

Um exemplo: temos um pacote TCP e quando vamos fazer um pacote
Ethernet usamos o pacote TCP inteiro como dados para o pacote Ethernet.

\section*{Questão 25}\

Um protocolo são vários conjuntos de ações que são estabelecidas para atingir
um objetivo, em outras palavras são comportamentos acordados necessários para
uma troca de informações.

\section*{Questão 26}

\begin{itemize}
    \item \textbf{Atraso de transmissão}:
        é o tempo necessário para transmitir cada bit do pacote.
    \item \textbf{Atraso de fila}:
        é o tempo que o pacote está em espera em fila antes de ser processado.
    \item \textbf{Atraso de processamento}:
        é o tempo necessário para ler o cabeçalho do pacote,
        checar integridade dos dados,
        e determinar para onde deve ser encaminhado.
    \item \textbf{Atraso de propagação}:
        é o tempo que leva para o pacote viajar de um nó para outro da rede.
\end{itemize}

\section*{Questão 27}\

Uma possibilidade é ter muitas consultas ao banco de dados
em um pequeno espaço de tempo,
gerando um ``gargalo''
(por atraso de fila e/ou processamento)
nos últimos nós da loja,
que estão conectados à fibra ótica.
Uma solução para esse problema seria adicionar mais nós
conectados à fibra ótica.

Outro motivo seria a grande distância
entre a loja e o banco de dados,
levando a altos atrasos de propagação.
Uma solução seria mover o banco de dados
para um local mais próximo da loja.

Uma maneira relativamente fácil de saber se é uma ou outra
é esperar um momento onde a rede não está sendo usada
e pingar o banco de dados.
Se o ping demorar muito,
mesmo com a rede sem uso,
então há um grande atraso de propagação.

\end{document}
