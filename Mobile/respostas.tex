\documentclass{article}

\usepackage[brazilian]{babel}

\usepackage{verbatim}

\renewcommand{\v}[1]{{}}

\begin{document}
\title{Mobile}
\author{Hashimoto}
\date{}
\maketitle

\section*{Questão 1}

Sistema Opecional Android é um sistema operacional adaptado do linux para
rodar aplicações no "estilo java" em celulares.

\section*{Questão 2}

Dalvik Virtual Machine é uma virtual usada para rodar programas em Androids.
DVM possui seu próprio bytecode e é otimizada (Just In Time)
para o ambiente móvel.

\section*{Questão 3}

Android Runtime usa uma técnica de compilação de código AOT (Ahead Of Time),
levando a melhor desempenho e economia de bateria.

\section*{Questão 4}

Os passos de construção do apk são:
\begin{enumerate}
    \item Compilar .java para .class
    \item Traduzir .class para .dex
    \item Juntar .dex para construir .apk
    \item Assinar o .apk
    \item Alinhar o .apk
\end{enumerate}

\section*{Questão 5}

Smali é um assembly language para bytecodes da DVM.

\section*{Questão 6}

\texttt{\
method(I[[\v]\v]ILjava/lang/String;[\v]Ljava/lang/Object;)Ljava/lang/String;
}

\section*{Questão 7}

Android Manifest é o arquivo que descreve o app,
mostra informações importantes, como:
\begin{itemize}
    \item Nome do pacote
    \item Permissões de acesso
    \item Entry point (onde o app começa)
    \item Nome do pacote
\end{itemize}

\section*{Questão 8}
\ 

A)
\begin{verbatim}
<activity android:label="@string/app_name" android:name="com.android.
insecurebankv2.LoginActivity">
    <intent-filter>
        <action android:name="android.intent.action.MAIN"/>
        <category android:name="android.intent.category.LAUNCHER"/>
    </intent-filter>
</activity>
\end{verbatim}

B)
Permissões de Risco:
\begin{itemize}
    \item Write External Storage
    \item Send SMS
    \item Use Credentials
    \item Get Accounts
    \item Read Profile
    \item Read Contacts
    \item Read External Storage
    \item Read Call Log
    \item Access Coarse Location
\end{itemize}

Permissões normais:
\begin{itemize}
    \item Internet
    \item Read Phone State
    \item Access Network State
\end{itemize}

C)
\verb|com.android.insecurebankv2|

D)
Pode fazer backup e é debuggable.
\begin{verbatim}
<application android:allowBackup="true" android:debuggable="true"
    android:icon="@mipmap/ic_launcher" android:label="@string/app_name"
    android:theme="@android:style/Theme.Holo.Light.DarkActionBar">
\end{verbatim}

\section*{Questão 9}

Intents são uma forma de ``chamar'' outros componentes
para fazer alguma coisa, por exemplo tirar/escolher uma foto/vídeo.

\section*{Questão 10}

Intent Explícito especifica o componente que vai ser chamado.
Já o Intent Implícito apenas informa o que é para ser feito
(e o SO ``escolhe alguém para fazer'').

\section*{Questão 11}

Intent-filters são especificações para o SO
dizendo as ``ações'' que um componente pode ser chamado
para fazer.

\section*{Questão 12}
\ 

A)
Implícito. O componente não é informado.

B)
Implícito. O componente não é informado.

C)
Explícito. O componente é informado e é \texttt{ReceiverActivity.class}.

\end{document}
