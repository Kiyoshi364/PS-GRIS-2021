\documentclass{article}

\usepackage[brazilian]{babel}
\usepackage{amsmath}
\usepackage{verbatim}

\begin{document}
\title{WriteUp Lattices}
\author{Hashimoto}
\date{}
\maketitle

\section{Vectors}

O texto explica aritmética básica de Vetores
e pede para calcular:
\[
    3 * ( 2 * v - w ) \cdot 2 * u
\]
dado que 
\(
    v = (2, 6, 3)
\) \quad \(
    w = (1, 0, 0)
\) \quad \(
    u = (7, 7, 2)
\).

Esse problema é introdutório,
dá para fazer ``na mão'':

\begin{gather*}
    3 * ( 2 * v - w ) \cdot 2 * u \\
    3 * ( 2 * (2, 6, 3) - (1, 0, 0) ) \cdot 2 * (7, 7, 2) \\
    3 * ( (4, 12, 6) - (1, 0, 0) ) \cdot (14, 14, 4) \\
    3 * (3, 12, 6) \cdot (14, 14, 4) \\
    (9, 36, 18) \cdot (14, 14, 4) \\
    9 \times 14 + 36 \times 14 + 18 \times 4 \\
    126 + 504 + 72 \\
    702
\end{gather*}

GG!

\section{Size and Basis}

O texto explica coisas sobre bases e tamanho/magnitude do vetor
e pede para calcular o tamanho de
\(
    (4, 6, 2, 5)
\)

Novamente, dá para fazer na mão:

\begin{gather*}
    \| (4, 6, 2, 5) \| \\
    \sqrt{ 4 \times 4 + 6 \times 6 + 2 \times 2 + 5 \times 5 } \\
    \sqrt{ 16 + 36 + 4 + 25 } \\
    \sqrt{ 81 }
    9
\end{gather*}

GG!

\section{Gram Schmidt}

O texto dá o algoritmo de \emph{Gram-Schmidt},
mas não normaliza, só torna os vetores ortogonais entre si
(mudei um pouco o formato):
\texttt{ \\
    For (\(i := 0\); \(i < n\); \(i\ +\!\!= 1\)) do \\
    . \(u_i := v_i\) \\
    . For (\(j := 0\); \(j < i\); \(j\ +\!\!= 1\)) do \\
    . . \(proj_{v_i \to u_j} := \frac{ v_i \cdot u_j }{ \|u_j\| } * u_j\) \\
    . . \(u_i = u_i - proj_{v_i \to u_j} \) \\
    . End For \\
    End For 
}

O objetivo é achar o \(u_4[1]\)
(arredondado para 5 casa decimais)
com essa entrada:
\[
    v_1 = (4, 1, 3, -1), \quad
    v_2 = (2, 1, -3, 4), \quad
    v_3 = (1, 0, -2, 7), \quad
    v_4 = (6, 2, 9, -5)
\]

Dessa vez fazer na mão vai ser meio chato,
então fiz uma pequena biblioteca em \texttt{zig}
(está em \texttt{``src/vector.zig''}).

Fiz um programa com a entrada hard-coded
(\texttt{gran-schmidt.zig}),
e essa é a saída:
\begin{verbatim}
[
    { 4.0e+00, 1.0e+00, 3.0e+00, -1.0e+00 },
    { 2.5925925925925926e+00, 1.1481481481481481e+00, -2.5555555555555554e+00,
3.851851851851852e+00 },
    { -7.229219143576826e-01, -1.0201511335012596e+00, 2.012594458438287e+00,
2.125944584382871e+00 },
    { -3.619189659458133e-01, 9.161073825503352e-01, 2.1488938603032537e-01,
1.1309967685806699e-01 }
]
\end{verbatim}

Nosso \(u_4\) está na última linha, \(u_4[1]\) é
\begin{align*}
    & 9.161073825503352e-01 \\
    & 0.9161073825503352 \\
    & 0.91611
\end{align*}

GG!

\section{What's a Lattice}

O texto explica o que é uma \emph{Lattice},
um \emph{Dominio Fundamental} de uma base
e como se calcula o volume de um Domínio Fundamental dessa base.
Então pede para calcular o volume do Domínio Fundamental dessa base:
\[
    v_1 = (6, 2, -3), \quad
    v_2 = (5, 1, 4), \quad
    v_3 = (2, 7, 1)
\]

Esse dá para fazer na mão,
mas é mais fácil escrever um programa que faz as contas para mim.
Está aqui (em \texttt{``src/lattice.zig''}) a saída:
\begin{verbatim}
    -2.55e+02
\end{verbatim}

Ou seja, \(-255\).

NotGG \(:(\)

Ah, o ``volume não pode ser negativo''.
Então... \(255\).

GG!

\section{Gaussian Reduction}

O texto cita os Problemas de procura
do \emph{Menor Vetor} e do \emph{Vetor Mais Próximo}
em uma Lattice.

A \emph{Redução de Gauss em uma Lattice} é um algorítmo
usado para resolver o Problema do Menor Vetor.
\texttt{ \\
    \( v_1, v_2 \) \\
    While (\(true\)) do \\
    . If ( \(\|v_2\| < \|v_1\|\) ) do \\
    . . \(tmp := v_1\) \\
    . . \(v_1 := v_2\) \\
    . . \(v_2 := tmp\) \\
    . End If \\
    . \# Isso é uma divisão inteira \\
    . \(m := floor\left(\frac{ v_1 \cdot v_2 }{ v_1 \cdot v_2 }\right) \) \\
    . If (\(m = 0\)) do \\
    . . return \(v_1\), \(v_2\) \\
    . End If \\
    . \(v_2 := v_2 - m * v_1\) \\
    End While
}

Temos que aplicar o algoritmo para
\[
    v_1 = (846835985, 9834798552), \quad
    v_2 = (87502093, 123094980)
\]

A saída \texttt{gaussian-reduction.zig} é:
\begin{verbatim}
    7410790865146821
\end{verbatim}

GG!

\section{Find the Lattice}

O texto me dá um algoritmo de geração de chaves e
encriptação/decriptação, a chave pública e a cifra,
temos que achar a chave privada e
decriptar a cifra para achar a mensagem original.

Algoritmo de geração de chaves é:
\texttt{ \\
    pub.x := primo(512) \\
    limiteMax := pub.x / 2 \\
    limiteMin := pub.x / 4 \\
    priv.x := random(2, limiteMax) \\
    While ($true$) do \\
    . priv.y := randim(limiteMin, limiteMax) \\
    . if (\(\)mdc(priv.x, priv.y)\( = 1\)) do \\
    . . break While \\
    . end If \\
    end While \\
    temp := inverseMod(priv.x, pub.x) \\
    pub.y = \((temp * priv.y) \% pub.x\) \\
    return pub priv
}

Algoritmo de encriptação é:
\texttt{ \\
    m := mensagem \\
    r := random(2, pub.x / 2) \\
    cifra := \((r * pub.y + m) \% pub.x\) \\
    return cifra
}

Algoritmo de decriptação é:
\texttt{ \\
    c := cifra \\
    temp := \((priv.x * c) \% pub.x\) \\
    mensagem := \((temp * \)inverseMod(priv.x, priv.y)\() \% priv.y\) \\
    return mensagem
}

As informações dadas são:
\begin{gather*}
    pub.x = \\
    7638232120454925879231554234011
    8423476410178882190211753042173 \\
    5871587863618325243345489649067
    7496516149889316745664606749499 \\
    241420160898019203925115292257 \\
    pub.y = \\
    2163268902194560093843693572170
    1997075017877974979984634621295 \\
    9223997358146265162297828263751
    3865274199374452805292639586264 \\
    791317439029535926401109074800 \\
    cifra = \\
    560569649525372066414288195690
    862430757067185847748211965743 \\
    616366366384473116903568234497
    428637904912373335600912567192 \\
    428031253275524116226726912348
    6523
\end{gather*}

Por ler o algoritmo de geração de chaves sabemos que:
\begin{itemize}
    \item[-] \texttt{pub.x} é um primo de 512 bits.
    \item[-] \( 2 \le \texttt{priv.x} < \frac{\texttt{pub.x}}{2} \)
    \item[-] \( \frac{\texttt{pub.x}}{4} \le \texttt{priv.y} < \frac{\texttt{pub.x}}{2} \)
    \item[-] \texttt{priv.x}, \texttt{priv.y} são primos entre si
    \item[-] \( \texttt{priv.x} * temp = 1 \ mod \texttt{pub.x} \)
    \item[-] \( \texttt{pub.y} = \texttt{priv.y} * temp) \ mod \texttt{pub.x} \)
    \item[+] \( \texttt{priv.x} * \texttt{pub.y} = \texttt{priv.y} \ mod \texttt{pub.x} \)
\end{itemize}

Por ler o algoritmo de encriptação sabemos que:
\begin{itemize}
    \item[-] \( 2 \le r < \frac{\texttt{pub.x}}{2} \)
    \item[-] \( \texttt{cifra} = r * \texttt{pub.y} + \texttt{m} \ mod \texttt{pub.x} \)
\end{itemize}

Por ler o algoritmo de decriptação sabemos que:
\begin{itemize}
    \item[-] \( temp = \texttt{priv.x} * \texttt{cifra} \ mod \texttt{pub.x} \)
    \item[-] \( \texttt{priv.x} * inv = 1 \ mod \texttt{priv.y} \)
    \item[-] \( \texttt{m} = temp * inv \ mod \texttt{priv.y} \)
    \item[+] \( \texttt{priv.x} * \texttt{m} = temp \ mod \texttt{priv.y} \)
\end{itemize}

No fim, \emph{graças a ajudas univesitárias},
só precisamos disso
\begin{itemize}
    \item \( \texttt{priv.x} * \texttt{pub.y} = \texttt{priv.y} \ mod \texttt{pub.x} \)
\end{itemize}
\begin{gather*}
    \texttt{priv.x} * \texttt{pub.y} = \texttt{priv.y} \ mod \texttt{pub.x} \\
    \texttt{priv.x} * \texttt{pub.y} - k * \texttt{pub.x} = \texttt{priv.y} \\
    \texttt{priv.x} (\texttt{pub.y}, 1) -  k (\texttt{pub.x}, 1) =
        (\texttt{priv.y}, \texttt{priv.x} - k)
\end{gather*}

Se usarmos \( \{ (\texttt{pub.y}, 1) , (\texttt{pub.x}, 1) \} \)
como entrada para Redução de Gauss,
vamos saber \emph{magicamente} que
um dos retornos \( (x_1, y_1), (x_2, y_2) \) é o \texttt{priv.y}.

Com o nosso \texttt{priv.y} (tentativa e erro com 4 possibilidades)
podemos calcular o \texttt{priv.x}:
\[
    \texttt{priv.x} =
    \texttt{priv.y} * \texttt{inverseMod(pub.y, pub.x)} \ mod \texttt{pub.x}
\]

Isso foi implementado em \texttt{where.py}
\textit{(infelizmente zig não suporta números tão grande}.

\begin{verbatim}
0 b'crypto{Gauss_lattice_attack!}'
1 b"u\xc9F\xfa\x91'\xb6[Y\xf6\xbb>\x1e\xb1\r\x8eN\xba\xe1\xf7\x90}mu\\u\xb3
\xb6\xa9\x00\x15t"
2 b'\x13C#d\x1cd@\xe60\xfdP\x19\xe8\x1d\xe8\x8f\xf0\x0br\xefV\xb3b\x11\xe6\x02
\xe4\xf3*3\xb2\xc9'
3 b'\x00'
\end{verbatim}

Está lá na primeira tentativa \verb.crypto{Gauss_lattice_attack!}.

GG!

\end{document}
